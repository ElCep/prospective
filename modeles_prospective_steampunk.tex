% 	Name		:: 	sthlm Beamer Theme  HEAVILY based on the hsrmbeamer theme (Benjamin Weiss)
%	Author		:: 	Mark Hendry Olson (mark@hendryolson.com)
%	Created		::	2013-07-31
%	Updated		::	June 18, 2015 at 08:45
%	Version		:: 	1.0.2
%	Email		:: 	hendryolson@gmail.com
%	Website		:: 	http://v42.com
%
% 	License		:: 	This file may be distributed and/or modified under the
%                  	GNU Public License.
%
%	Description	::	This presentation is a demonstration of the sthlm beamer
%					theme, which is HEAVILY based on the HSRM beamer theme created by Benjamin Weiss
%					(benjamin.weiss@student.hs-rm.de), which can be found on GitHub
%					<https://github.com/hsrmbeamertheme/hsrmbeamertheme>.


%-=-=-=-=-=-=-=-=-=-=-=-=-=-=-=-=-=-=-=-=-=-=-=-=
%
%        LOADING DOCUMENT
%
%-=-=-=-=-=-=-=-=-=-=-=-=-=-=-=-=-=-=-=-=-=-=-=-=

\documentclass[newPxFont]{beamer}
\usetheme{sthlm}
%\usecolortheme{sthlmv42}

%-=-=-=-=-=-=-=-=-=-=-=-=-=-=-=-=-=-=-=-=-=-=-=-=
%        LOADING PACKAGES
%-=-=-=-=-=-=-=-=-=-=-=-=-=-=-=-=-=-=-=-=-=-=-=-=
\usepackage[utf8]{inputenc}
\usepackage[T1]{fontenc}

\usepackage{chronology}
\usepackage{subfigure}
\usepackage{wrapfig} %floting figure
\usepackage{color} %color in white text in chronologie


%\definecolor{Blue}{rgb}{0,111,174}

\renewcommand{\event}[3][e]{%
  \pgfmathsetlength\xstop{(#2-\theyearstart)*\unit}%
  \ifx #1e%
    \draw[fill=black,draw=none,opacity=0.5]%
      (\xstop, 0) circle (.2\unit)%
      node[opacity=1,rotate=45,right=.2\unit] {#3};%
  \else%
    \pgfmathsetlength\xstart{(#1-\theyearstart)*\unit}%
    \draw[fill=black,draw=none,opacity=0.5,rounded corners=.1\unit]%
      (\xstart,-.1\unit) rectangle%
      node[opacity=1,rotate=45,right=.2\unit] {#3} (\xstop,.1\unit);%
  \fi}%

\DeclareUnicodeCharacter{00A0}{~}
%-=-=-=-=-=-=-=-=-=-=-=-=-=-=-=-=-=-=-=-=-=-=-=-=
%        BEAMER OPTIONS
%-=-=-=-=-=-=-=-=-=-=-=-=-=-=-=-=-=-=-=-=-=-=-=-=

%\setbeameroption{show notes}

%-=-=-=-=-=-=-=-=-=-=-=-=-=-=-=-=-=-=-=-=-=-=-=-=
%
%	PRESENTATION INFORMATION
%
%-=-=-=-=-=-=-=-=-=-=-=-=-=-=-=-=-=-=-=-=-=-=-=-=

\title{Modelisation spatiale \\et prospective territoriale}
\subtitle{sont-elles soluble dans le Steampunk ?}
%\date{\small{\jobname}}
%\date{\today}
\date{14 Novembre 2018}
\author{\texttt{Etienne DELAY}}
\institute{UR GREEN}

\hypersetup{
pdfauthor = {Etienne DELAY},
pdfsubject = {Séminaire " Prospective territoriale et modelisation "},
pdfkeywords = {modelisation, simulation, prospective},
pdfmoddate= {D:\pdfdate},
pdfcreator = {}
}

\begin{document}

%-=-=-=-=-=-=-=-=-=-=-=-=-=-=-=-=-=-=-=-=-=-=-=-=
%
%	TITLE PAGE
%
%-=-=-=-=-=-=-=-=-=-=-=-=-=-=-=-=-=-=-=-=-=-=-=-=

\maketitle

%\begin{frame}[plain]
%	\titlepage
%\end{frame}

%-=-=-=-=-=-=-=-=-=-=-=-=-=-=-=-=-=-=-=-=-=-=-=-=
%
%	TABLE OF CONTENTS: OVERVIEW
%
%-=-=-=-=-=-=-=-=-=-=-=-=-=-=-=-=-=-=-=-=-=-=-=-=
%\section*{Overview}
%\begin{frame}{Overview}
%% For longer presentations use hideallsubsections option
%\tableofcontents[hideallsubsections]
%\end{frame}

\section{Le Steampunk ?}

%-=-=-=-=-=-=-=-=-=-=-=-=-=-=-=-=-=-=-=-=-=-=-=-=
%	FRAME: LE STEAMPUNK, LE MYSTERE D'UN TITRE
%-=-=-=-=-=-=-=-=-=-=-=-=-=-=-=-=-=-=-=-=-=-=-=-=

\begin{frame}[c]{Lever les ambiguités : Steampunk ? }
\vspace{-1cm}
Proto-science-fictions, mettant en scène des pionniers scientifiques uchronique dans des decores victorien.
\begin{figure}
  \includegraphics[height=6cm]{img/a_steampunk_car.jpg}
\end{figure}
\end{frame}

\begin{frame}[c]{Modelisation, prospective et Steampunk?}
\vspace{-1cm}
Reviens à questionner le status de l'un par rapport à l'autre
\begin{figure}
  \includegraphics[height=6cm]{img/a_Tom-Gauld-jetpack.jpg}
\end{figure}
\end{frame}

\begin{frame}[c]{Comment cela va-t-il se passer ?}
\vspace{-1cm}
\begin{itemize}
  \item Le status de la modélisation spatialement expliciter
  \item La prospective territorialement
  \item Un exemple d'hybridation réussie ?
\end{itemize}

\begin{figure}
  \includegraphics[height=4cm]{img/a_Tom-Gauld-walk.jpg}
\end{figure}
\end{frame}

\section{Epistemologie de la\\ modelisation : le domaine}

%-=-=-=-=-=-=-=-=-=-=-=-=-=-=-=-=-=-=-=-=-=-=-=-=
%	FRAME: EPISTEMOLOGIE DE LA MODELISATION
%-=-=-=-=-=-=-=-=-=-=-=-=-=-=-=-=-=-=-=-=-=-=-=-=

\begin{frame}[c]{Du Système au modèle, tentatives de thérorisation}
\vspace{-1em}
\begin{quote}
  \enquote{le terme de \textbf{modèle} a la même signification que celui de concept ou d'hypothèse ou d'analogie [...], un modèle est une abstraction qui simplifie le système réel étudié [...] pour se focaliser sur les aspects qui intéressent le modélisateur et qui définissent les problématiques du modèle.}
\end{quote}

\hspace*{\fill}\textsc{Coquillard et Hill 1997, p.7}
\vspace{-0.5em}
\begin{figure}
 	\centering
 		\subfigure{\includegraphics[height=3cm]{img/a_descartes.jpg}} \hspace{0.2em} %%descartes
    \subfigure{\includegraphics[height=3cm]{img/a_newton.jpg}} \hspace{0.2em} %% newton
 		\subfigure{\includegraphics[height=3cm]{img/a_pasteur.jpg}} \hspace{0.2em} %%pasteur
    \subfigure{\includegraphics[height=3cm]{img/a_thom.png}} %%thom
\end{figure}
\end{frame}

\begin{frame}[c]{Du Système au modèle, tentatives de thérorisation}
\vspace{-1em}
\begin{quote}
  \enquote{<<la \textbf{théorisation} [...] est liée à la possibilité de plonger le réel dans un virtuel imaginaire, doté de propriétés génératives, qui permettent de faire des prévisions.>>}
\end{quote}
\hspace*{\fill}\textsc{Thom 2009, p. 91}
\vspace{-0.5em}
\begin{figure}
 	\centering
 		\subfigure{\includegraphics[height=3cm]{img/a_descartes.jpg}}\hspace{0.2em} %%descartes
    \subfigure{\includegraphics[height=3cm]{img/a_newton.jpg}}\hspace{0.2em} %% newton
 		\subfigure{\includegraphics[height=3cm]{img/a_pasteur.jpg}}\hspace{0.2em} %%pasteur
    \subfigure{\includegraphics[height=3cm]{img/a_thom.png}} %%thom
\end{figure}
\end{frame}

\begin{frame}[c]{Des modèles pour la recherche de formes}
\vspace{-1em}
\begin{quote}
  \enquote{Peut-on, dans un paysage de phénomènes, reconnaître un objet ou une chose si l'on n'en a pas au préalable le concept ? C'est aussi simple que ça. Si l'on n'a pas le concept d'un objet, on ne le reconnaîtra pas. [...] La possibilité de reconnaître un être en général, une entité dans un paysage empirique, est toujours à mon avis subordonnée à une conceptualisation}
\end{quote}
\hspace*{\fill}\textsc{Thom 2009, p.93}
\vspace{-0.5em}
\begin{figure}
 \includegraphics[height=3cm]{img/a_rorschach.png}
\end{figure}
\end{frame}

\section{Epistemologie de la\\ modelisation : controverse}
%-=-=-=-=-=-=-=-=-=-=-=-=-=-=-=-=-=-=-=-=-=-=-=-=
%	FRAME: CONTROVERSE
%-=-=-=-=-=-=-=-=-=-=-=-=-=-=-=-=-=-=-=-=-=-=-=-=
\begin{frame}[c]{Le réductionnisme}
\vspace{-1em}
\begin{quote}
  \enquote{<<Consiste à fragmenter des systèmes complexes en éléments plus simples à étudier.>>}
\end{quote}
\hspace*{\fill}\textsc{Wiktionnaire 2018}
\vspace{-0.5em}
\begin{figure}
 \includegraphics[height=3cm]{img/a_turc_automaton.jpg}
\end{figure}
\end{frame}

\begin{frame}[c]{Le holisme}
\vspace{-1em}
\begin{quote}
  \enquote{<<consiste à considérer les phénomènes individuels ou particuliers comme faisant partie de la totalité dans laquelle ils s’inscrivent.>>}
\end{quote}
\hspace*{\fill}\textsc{Wiktionnaire 2018}
\vspace{-0.5em}
\begin{figure}
 \includegraphics[height=3cm]{img/a_wine.jpeg}
\end{figure}
\end{frame}

\begin{frame}[c]{Une $3^{\`eme}$ voie : la \textit{complexité}}
\vspace{-1em}
\begin{quote}
  \enquote{TODO}
\end{quote}
\hspace*{\fill}\textsc{Thom 2009, p.93}
\vspace{-0.5em}
\begin{figure}
 \includegraphics[angle=90,height=5cm]{img/a_rule_wolfram.png}
\end{figure}
\end{frame}

\begin{frame}[c]{Gardes fous}
\vspace{-1em}
\begin{quote}
  \enquote{<<Cela n'implique pas que l'on prenne définitivement parti pour le holisme (\textsc{Valade 2001}), contre l'individualisme méthodologique (\textsc{Boudon 2004}), mais cela implique qu'on garde l'un et l'autre en mémoire comme gardes fous de l'autre>>}
\end{quote}
\hspace*{\fill}\textsc{Pumain 2003 p.26}
\vspace{-0.5em}
\begin{figure}
 \includegraphics[height=3cm]{img/a_Funambule.jpg}
\end{figure}
\end{frame}

\section{Epistemologie de la\\ modelisation : simulation}
%-=-=-=-=-=-=-=-=-=-=-=-=-=-=-=-=-=-=-=-=-=-=-=-=
%	FRAME: CONTROVERSE
%-=-=-=-=-=-=-=-=-=-=-=-=-=-=-=-=-=-=-=-=-=-=-=-=

\begin{frame}[c]{Modelisation et complexité}
\vspace{-2em}
Les théories des systèmes complexe on nécéssité
\begin{itemize}
  \item de changer de point et d'approche sur les problèmes posés
  \item de reconnaitre l'importance des phénomènes locaux sur des complortement à des échelles supra.
  \item de s'intéressé aux questions d'émergence / auto-organisation
\end{itemize}
\begin{figure}
 \includegraphics[height=6cm]{img/a_crowd.jpg}
\end{figure}
\end{frame}

\begin{frame}[c]{Philigenie des approches de simulation}
\vspace{-2em}
\begin{figure}
 \includegraphics[height=6cm]{img/a_troitzsch_1997.png}
\end{figure}
\vspace{-0.8em}
\small{Le développement de l'approche de simulation contemporaine en sciences sociales (d'après \textsc{Troitzsch} 1997). La partie grise représente les modèles à base d'équation, la partie blanche les modèles à base d'objets, d'événements, ou d'agents}.
\end{frame}

\begin{frame}[c]{Les automates cellulaire}
\vspace{-2em}
\begin{itemize}
  \item un état
  \item un nombre d'état fini
  \item un ensemble de règles de changement d'état
\end{itemize}
\begin{figure}
 \includegraphics[height=4cm]{img/a_gameofliferules11.jpg}
\end{figure}
\vspace{-0.8em}
\end{frame}

\begin{frame}[c]{Les systèmes multi-agents}
\vspace{-2em}
\small
\begin{quote}
  "[...] une entité physique ou virtuelle :
    \begin{itemize}
      \item qui est capable d'agir dans un environnement ;
      \item qui peut communiquer directement avec d'autres agents ;
      \item qui est mue par un ensemble de tendances;
      \item qui est capable de percevoir (mais de manière limitée) son environnement ;
      \item qui ne dispose que d'une représentation partielle de cet environnement (et éventuellement aucune) ;
      \item dont le comportement tend à satisfaire ses objectifs, en tenant compte des ressources et des compétences dont elle dispose, et en fonction de sa perception et des communications qu'elle reçoit".
    \end{itemize}
\end{quote}
\hspace*{\fill}\textsc{Ferber 1995, p.13}
% \begin{figure}
%  \includegraphics[height=4cm]{img/a_jennings_1999.png}
% \end{figure}
% \vspace{-0.8em}
\end{frame}

\begin{frame}[c]{La place de l'empirisme dans la simulation}
\vspace{-2em}
\begin{figure}
 \includegraphics[height=7cm]{img/a_theory_cassili.png}
\end{figure}
\hspace*{\fill}\textsc{(Tubaro et Casilli, 2010)}
\end{frame}

%-=-=-=-=-=-=-=-=-=-=-=-=-=-=-=-=-=-=-=-=-=-=-=-=
%
%	SECTION: PROSPECTIVE
%
%-=-=-=-=-=-=-=-=-=-=-=-=-=-=-=-=-=-=-=-=-=-=-=-=
\section{La prospective}

\begin{frame}[c]{La prospective : au passé}
\vspace{-2em}
Issue du monde de l'entreprise au moment de la dissociation des responsablitées
\begin{itemize}
  \item stratégiques (fixer les objectifs)
  \item tactiques (moyen pour y parvenir)
\end{itemize}
\textbf{objectif} : Une vision a long terme pour ne pas se laisser perturber par les ajustements que nécéssite la réalité.
\begin{figure}
 \includegraphics[height=4cm]{img/a_gauld_tom_new_scientist.jpg}
\end{figure}
\end{frame}

\begin{frame}[c]{Problèmes perniceux et sciences post-normal}
\vspace{-2em}
Une démarche prospective essai d'anticiper les "problèmes pernicieux" :  Difficile ou impossible à resoudre en raison d'exigences incomplètes, contradictoires et changeantes qui sont souvent difficiles à reconnaître ().

Le recoure à la participation des partie-prenantes quand les processus sont complexes, les faits incertains, les valeurs discutées et les enjeux élevés (\textsc{Funtowics et Ravetz 1993})

\begin{figure}
 \includegraphics[height=4cm]{img/a_dilbert-climate-science.png}
\end{figure}
\end{frame}

\begin{frame}[c]{Sur les épaules des géants}
\vspace{-2em}
\begin{quote}
  \enquote{<<Ce n'est donc pas seulement le passé qui explique l'avenir, mais aussi l'image du futur qui s'imprime dans le présent>>}
  \hspace*{\fill}\textsc{Godet 1985 p.29}
\end{quote}
La prospective n'a d'intérêt qu'a partir du moment où les acteurs s'en saissisent pour mener à bien des projets sur le temps long. (\textsc{M. Sebillote, Aigrain}, \textit{et al.} 2003, p. 330)
\begin{figure}
 \includegraphics[height=4cm]{img/a_thom_gauld_futureMachine.jpg}
\end{figure}
\end{frame}

%-=-=-=-=-=-=-=-=-=-=-=-=-=-=-=-=-=-=-=-=-=-=-=-=
%
%	SECTION: CONCLUSION
%
%-=-=-=-=-=-=-=-=-=-=-=-=-=-=-=-=-=-=-=-=-=-=-=-=
\section{Conclusion}

%-=-=-=-=-=-=-=-=-=-=-=-=-=-=-=-=-=-=-=-=-=-=-=-=
%	FRAME: CONCLUSION
%-=-=-=-=-=-=-=-=-=-=-=-=-=-=-=-=-=-=-=-=-=-=-=-=
\begin{frame}[c]{Conclusion}
\vspace{-2em}
L'espace peut être considéré comme une production sociale (\textsc{Auriac, 2000, p.174}), et que celle-ci est robuste aux changements d'échelle
\end{frame}

{
%\usebackgroundtemplate{\includegraphics[width=1.05\paperwidth]{img/bw_fin.jpg}}%
\usebackgroundtemplate{\includegraphics[width=1.05\paperwidth]{img/fin_diapo}}%
\begin{frame}
  \begin{minipage}[t][.8\textheight]{\textwidth}
    %\color{\cnGrey}{\LARGE{Merci de votre attention}}

    \vfill

    %\hfill \small{Crédit photo : Thomas Misnyovszki sur Flick'r}
    \hfill \color{\cnGrey}{\LARGE{Merci de votre attention}}

    \hfill \small{Atelier participatif : Sakal 2017}
  \end{minipage}

\end{frame}
}

%%-=-=-=-=-=-=-=-=-=-=-=-=-=-=-=-=-=-=-=-=-=-=-=-=================
%%	FIN DU DIAPORAMA OFFICIEL
%%-=-=-=-=-=-=-=-=-=-=-=-=-=-=-=-=-=-=-=-=-=-=-=-=================


\end{document}
